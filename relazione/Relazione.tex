\documentclass[12pt]{article}
\usepackage{titling}
\usepackage{graphicx}
\usepackage{caption}
\usepackage{helvet}
\usepackage{hyperref}
\usepackage{bookmark}
\usepackage{framed}
\usepackage{listings}
\usepackage{mdframed}
\usepackage{sectsty}
\usepackage{tikz}
\usepackage{amssymb}
\newcommand{\abs}[1]{\left|#1\right|}

\renewcommand{\lstlistingname}{}

% Define font size and family for sections
\sectionfont{\fontsize{12}{14}\selectfont\renewcommand{\familydefault}{\sfdefault}}
\subsectionfont{\fontsize{12}{14}\selectfont\renewcommand{\familydefault}{\sfdefault}}
\subsubsectionfont{\fontsize{12}{14}\selectfont\renewcommand{\familydefault}{\sfdefault}}
\paragraphfont{\fontsize{12}{14}\selectfont\renewcommand{\familydefault}{\sfdefault}}
\subparagraphfont{\fontsize{12}{14}\selectfont\renewcommand{\familydefault}{\sfdefault}}
\renewcommand{\familydefault}{\rmdefault} % Reset default to Roman for title page

\usepackage[utf8]{inputenc}
\usepackage{amsmath}
\usepackage{amssymb}
\usepackage{hyperref}

\lstset{
    language=Matlab,
    basicstyle=\ttfamily\footnotesize,
    keywordstyle=\color{blue},
    commentstyle=\color{green},
    stringstyle=\color{red},
    frame=single,
    breaklines=true,
    keepspaces=true,
    showspaces=false,
    showstringspaces=false
}


\begin{document}

% Prima pagina
\begin{titlepage}
    \centering
    {\Huge \textbf{Università degli Studi di Padova} \par}
    \vspace{1cm}
    \begin{figure}[h!]
        \centering
        \includegraphics[width=0.5\textwidth]{Immagini/Logo_Università_Padova.png}
    \end{figure}
    \vspace{0.5cm}
    {\LARGE \textbf{Relazione per:} \par}
    {\Huge Ricerca Operativa \par}
    \vspace{0.5cm}
    {\Large \textit{\textbf{Per Simulare una:}} \par}
    {\Large Minimizzare il numero di turni nei giochi RPG \par}
    \vfill
    \textbf{Realizzata da:} \par
    Gabriele Di Pietro, matricola 2010000 \par
    \vspace{0.5cm}
    \textbf{GitHub Repository:} \par
    \texttt{\url{https://github.com/SerpenTaki/ProgettoRO-2025}}
\end{titlepage}
    % Change font and size for the rest of the document
    \renewcommand{\familydefault}{\sfdefault} % Set default to sans serif
    \fontsize{10}{12}\selectfont
    % Pagina introduzione
    \newpage
    \section{Introduzione}
    \subsection{Abstract}
    Si presenta un problema di ottimizzazione strategica nei giochi RPG dove il giocatore deve strategicamente comporre un team di personaggi per minimizzare il numero di turni impiegati per uccidere un eventuale boss.\\
    \subsection{Descrizione del problema}
    Per un gioco di ruolo il giocatore può scegliere di comporre un team di personaggi.
    Ogni team è formato da 3 personaggi scelti tra Cavalieri e Maghi. Questo significa che un team può essere formato da 3 Cavalieri, 3 Maghi o da un misto di entrambi. 
    La stamina e il mana sono statistiche separate ma condivise tra i personaggi; in generale
    il mana viene usato dai Maghi, mentre la stamina dai Cavalieri.
    I 3 personaggi possono decidere di attaccare durante lo stesso turno o di non attaccare per quello specifico turno per recuperare un certo quantitativo di mana e stamina.
    I Cavalieri, quando attaccano, possono decidere di usare la spada con 2 mani consumando il 75\% di stamina in più ma infliggendo il doppio dei danni. Mentre i Maghi, decidendo di non attaccare, potenziano l'attacco dei Cavalieri del 20\% per i 2 turni successivi. Non è possibile che 2 Maghi si riposino nello stesso turno. Si vuole minimizzare il numero di turni totali per uccidere un grande mostro con un determinato quantitativo di vita. Sapendo che:
    \begin{itemize}
        \item Un Boss ha un certo numero di punti vita;
        \item Nei turni di riposo il mago ripristina il 20$\%$ di mana consumato;
        \item Nei turni di riposo il cavaliere ripristina il 30$\%$ di stamina consumata;
    \end{itemize}

    \section{Modello}
    \subsection{Insiemi}
    \begin{itemize}
        \item $T$: Insieme dei turni possibili, con $t \in T \subseteq \mathbb{Z}^+$
        \item $K$: Insieme dei cavalieri, con $k \in K \subseteq \mathbb{Z}^+$ e $k = \{ 0,1,2,3 \}$
        \item $M$: Insieme dei maghi, con $m \in M \subseteq \mathbb{Z}^+$ e $m = 3- k$
    \end{itemize}
    \subsection{Parametri}
    \begin{itemize}
        \item $PV_{Boss}$: punti vita del Boss
        \item $DK$: danno che un cavaliere può infliggere
        \item $DM$: danno che un mago può infliggere
        \item $SKC$: consumo di stamina da parte di un cavaliere
        \item $MMC$: consumo di mana da parte di un mago
    \end{itemize}
   \subsection{Variabili}
   \begin{itemize}
    \item $S_t$: Stamina disponibile all'inizio del turno
    \item $M_t$: Mana disponibile all'inizio del turno
    \item $CS_{k,t}$: Stamina totale consumata dal cavaliere $k$ fino al turno $t$
    \item $CM_{m,t}$: Mana totale consumato dal mago $m$ fino al turno $t$
    \item $Boost_t$: Moltiplicatore di danno per i cavalieri (\textit{nel caso di riposo da parte del mago})
    \item $DPT_t$: Danno per turno, quindi il danno totale inflitto dal \textit{party}\footnote{Insieme di cavalieri e maghi che compongono il team.} al turno $t$
   \end{itemize}
   \subsubsection{Variabili decisionali}
   \begin{itemize}
    \item $x_{k,t} = \begin{cases} 
        1 & \text{se il cavaliere } k \text{ attacca al turno } t \\ 
        0 & \text{altrimenti} 
    \end{cases}$
    \item $y_{k,t} = \begin{cases} 
        1 & \text{se il cavaliere } k \text{ usa 2 mani per attaccare } \\ 
        0 & \text{altrimenti} 
    \end{cases}$
    \item $z_{m,t} = \begin{cases} 
        1 & \text{se il mago } m \text{ attacca nel turno } t \\ 
        0 & \text{altrimenti} 
    \end{cases}$
    \item $u_t = \begin{cases}
        1 & \text{se il boss è stato sconfitto nel turno } t \\
        0 & \text{altrimenti}
    \end{cases}$
    \end{itemize}
   \subsection{Funzione obiettivo}
    \begin{equation}
        \min \sum_{t \in T} t \cdot u_t
    \end{equation}
    L'obiettivo è quello di trovare il turno minimo in cui il boss viene sconfitto. Quindi sommiamo tutti i turni ma solo quello dove viene sconfitto non va a 0.\\
    \newpage
    \subsubsection{Vincoli}
    \begin{itemize}
        \item La stamina e il mana sono valori che si aggiornano in base al consumo (\textit{durante la fase di attacco}) o alla ricarica (\textit{durante la fase di riposo}).
    \end{itemize}
    \begin{tabular*}{\textwidth}{@{\extracolsep{\fill}} ll}
       $S_{t+1} = S_t - \sum_{k \in K} x_{k,t} \cdot SKC (1+ 0.75 \cdot y_{k,t}) + \sum_{k \in K} (1-x_{k,t}) \cdot 0.3 \cdot CS_{k,t-1}$ & $\forall k$ \\
       & \\
       $M_{t+1} = M_t - \sum_{m \in M} z_{m,t} \cdot MMC + \sum_{m \in M} (1-z_{m,t}) \cdot 0.2 \cdot CM_{m,t-1}$ & $\forall m$ \\
    \end{tabular*}
    \begin{itemize}
        \item Consumo cumulativo
    \end{itemize}
    \begin{tabular*}{\textwidth}{@{\extracolsep{\fill}} ll}
       $CS_{k,t} = CS_{k,t-1} + x_{k,t} \cdot SKC(1 + 0.75 y_{k,t})$ & $\forall k,t$ \\
        & \\
        $CM_{m,t} = CM_{m,t-1} + z_{m,t} \cdot MMC$ & $\forall m,t$ \\
     \end{tabular*}
    \begin{itemize}
        \item I Maghi decidendo di non attacare potenziano l'attacco dei Cavalieri per i 2 turni successivi
    \end{itemize}
    \begin{tabular*}{\textwidth}{@{\extracolsep{\fill}} ll}
        $Boost_{t} = 1 + 0.2(\sum_{m \in M}(1-z_{m,t-1})+\sum_{m \in M}(1+z_{m,t-2}))$ & $\forall t \geq 3$\\
    \end{tabular*}
    \begin{itemize}
        \item Vincoli sul danno e sulla condizione di vittoria
    \end{itemize}
    \begin{tabular*}{\textwidth}{@{\extracolsep{\fill}} ll}
        $DPT_t = (\sum_{k \in K} x_{k,t} \cdot DK \cdot (1 + y_{k,t}))\cdot Boost_{t} + \sum_{m \in M} z_{m,t} \cdot DM$ & $\forall t$\\
        & \\
        $\sum_{t \in T} \geq PV_{Boss} \cdot u_t$ & $\forall t$ con $\sum_{t \in T}u_t=1$ \\
    \end{tabular*}
    \begin{itemize}
        \item Vincolo di riposo dei maghi, in un turno solo 1 mago 
    \end{itemize}
    \begin{tabular*}{\textwidth}{@{\extracolsep{\fill}} ll}
        $\sum_{m \in M}(1-z_{m,t}) \leq 1$ & $\forall t$ \\
    \end{tabular*}
    \begin{itemize}
        \item Vincoli logici
    \end{itemize}
    \begin{tabular*}{\textwidth}{@{\extracolsep{\fill}} ll}
        $y_{k,t} \leq x_{k,t}$ & $\forall k,t$ \\
        & \\
        $\sum_{k \in K} x_{k,t} \cdot SKC(1 + 0.75 y_{k,t}) \leq S_t$ & $\forall t$ \\
        & \\
        $\sum_{m \in M} z_{m,t} \cdot MMC \leq M_t$ & $\forall t$ \\
        & \\
    \end{tabular*}\\
    \begin{table}[h] % Ambiente table corretto
        \centering
        \begin{tabular}{|c|c|}
            \hline
            \multicolumn{2}{|c|}{\textbf{Domini}} \\
            \hline
            $S_t \geq 0, \, S_t \in \mathbb{R}^+$ & $M_t \geq 0, \, M_t \in \mathbb{R}^+$ \\
            \hline
            $CS_{k,t} \geq 0, \, CS_{k,t} \in \mathbb{R}^+$ & $CM_{k,t} \geq 0, \, CM_{k,t} \in \mathbb{R}^+$ \\
            \hline
            $Boost_t \geq 1, \, Boost_t \in \mathbb{R}^+$ &  $DPT_t \geq 0, \, DPT_t \in \mathbb{R}^+$ \\
            \hline
            \multicolumn{2}{|c|}{$x_{k,t}, y_{k,t}, z_{m,t}, u_t \in \{0,1\}$ in $\mathbb{Z}^+$} \\
            \hline
        \end{tabular}
        \caption{Tabella dei Domini}
        \label{tab:domini}
    \end{table}
    
    \newpage
    \section{Implementazione}
    \subsection{File .mod}
    \subsection{File .run}
    \section{Primo scenario}
    \subsection{File .dat}
    \subsection{Risultati}
    \section{Secondo scenario}
    \subsection{File .dat}
    \subsection{Risultati}
\end{document}
