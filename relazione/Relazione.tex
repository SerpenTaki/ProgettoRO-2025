\documentclass[12pt]{article}
\usepackage{titling}
\usepackage{graphicx}
\usepackage{caption}
\usepackage{helvet}
\usepackage{hyperref}
\usepackage{bookmark}
\usepackage{framed}
\usepackage{listings}
\usepackage{mdframed}
\usepackage{sectsty}
\usepackage{tikz}
\newcommand{\abs}[1]{\left|#1\right|}

\renewcommand{\lstlistingname}{}

% Define font size and family for sections
\sectionfont{\fontsize{12}{14}\selectfont\renewcommand{\familydefault}{\sfdefault}}
\subsectionfont{\fontsize{12}{14}\selectfont\renewcommand{\familydefault}{\sfdefault}}
\subsubsectionfont{\fontsize{12}{14}\selectfont\renewcommand{\familydefault}{\sfdefault}}
\paragraphfont{\fontsize{12}{14}\selectfont\renewcommand{\familydefault}{\sfdefault}}
\subparagraphfont{\fontsize{12}{14}\selectfont\renewcommand{\familydefault}{\sfdefault}}
\renewcommand{\familydefault}{\rmdefault} % Reset default to Roman for title page

\usepackage[utf8]{inputenc}
\usepackage{amsmath}
\usepackage{hyperref}

\lstset{
    language=Matlab,
    basicstyle=\ttfamily\footnotesize,
    keywordstyle=\color{blue},
    commentstyle=\color{green},
    stringstyle=\color{red},
    frame=single,
    breaklines=true,
    keepspaces=true,
    showspaces=false,
    showstringspaces=false
}


\begin{document}

% Prima pagina
\begin{titlepage}
    \centering
    {\Huge \textbf{Università degli Studi di Padova} \par}
    \vspace{1cm}
    \begin{figure}[h!]
        \centering
        \includegraphics[width=0.5\textwidth]{Immagini/Logo_Università_Padova.png}
    \end{figure}
    \vspace{0.5cm}
    {\LARGE \textbf{Relazione per:} \par}
    {\Huge Ricerca Operativa \par}
    \vspace{0.5cm}
    {\Large \textit{\textbf{Per Simulare una:}} \par}
    {\Large Sessioni di sviluppo e formazione per un Azienda di videogiochi \par}
    \vfill
    \textbf{Realizzata da:} \par
    Gabriele Di Pietro, matricola 2010000 \par
    \vspace{0.5cm}
    \textbf{GitHub Repository:} \par
    \texttt{\url{https://github.com/SerpenTaki/ProgettoRO-2025}}
\end{titlepage}
    % Change font and size for the rest of the document
    \renewcommand{\familydefault}{\sfdefault} % Set default to sans serif
    \fontsize{10}{12}\selectfont
    % Pagina introduzione
    \newpage
    \section{Introduzione}
    \subsection{Abstract}
    Un'azienda videoludica organizza sessioni di sviluppo e formazione per il proprio team e per sviluppatori esterni. L'obiettivo del progetto è massimizzare il guadagno ricavato dalle sessioni, minimizzando i costi del personale qualificato.
    \subsection{Descrizione del problema}
    Durante ogni giorno lavorativo della settimana, l'azienda organizza sessioni di sviluppo e formazione in determinate fasce orarie. L'azienda offre corsi su vari aspetti della produzione videoludica, tra cui programmazione, game design e grafica. L'organizzazione settimanale delle sessioni è sempre la stessa ogni settimana.\\
    Per poter essere svolta, ogni sessione necessita di un esperto che la conduca. Un esperto può seguire un numero massimo di sessioni contemporaneamente, purché siano della stessa tipologia (programmazione, game design o grafica). Questo significa che se due sessioni di programmazione si svolgono nella stessa fascia oraria un esperto può gestirle entrambe, mentre se ad esempio una sessione di programmazione e di grafica si sovrappongono, dovranno essere assegnate a esperti diversi.\\
    Esiste anche una sessione avanzata per sviluppatori senior, della durata di 4 ore, che deve essere seguita dallo stesso esperto dedicato interamente a loro durante lo svolgimento della sessione in una giornata. Nel fine settimana, si possono organizzare sessioni speciali su progetti live, che devono essere seguite dallo stesso esperto per tutto il tempo.\\
    Si vuole massimizzare il guadagno derivato dalle sessioni, sapendo che:
    \begin{itemize}
        \item Un esperto ha un certo costo orario;
        \item Un esperto ha un numero massimo di ore settimanali che può mettere a disposizione;
        \item Lo svolgimento di una sessione di programmazione porta un determinato guadagno;
        \item Lo svolgimento di una sessione di game design porta un guadagno differente;
        \item Lo svolgimento di una sessione di grafica porta un altro tipo di guadagno;
        \item Per assicurare una formazione costante e strutturata agli sviluppatori senior, le sessioni avanzate devono essere svolte almeno un determinato numero di giorni alla settimana;
        \item Le sessioni avanzate vengono svolte in una certa fascia oraria, identica per ogni giorno della settimana in cui si tengono;
        \item Lo svolgimento delle 4 ore di una sessione avanzata porta un determinato tipo di guadagno;
        \item Le sessioni speciali del fine settimana hanno un determinato numero di ore;
        \item Le sessioni speciali, se svolte, portano un determinato guadagno.
    \end{itemize}
    \section{Modello}
    \subsection{Insiemi}
    \begin{itemize}
        \item $G$: Insieme dei giorni lavorativi della settimana;
        \item $H$: Insieme delle fasce orarie in cui si svolgono le sessioni;
        \item $E$: Insieme degli esperti;
    \end{itemize}
    \subsection{Parametri}
    \begin{itemize}
        \item $maxSessioni_e$: Numero massimo di sessioni che un esperto può seguire contemporaneamente $e \in E$;
        \item $C_e$: Costo orario di un esperto;
        \item $maxOreSettimanali$: Numero massimo di ore settimanali che un esperto può mettere a disposizione;
        \item $Programmazione_{g,o}$: Sessione individuale di programmazione che si svolge il giorno $g \in G$ nella fascia oraria $h \in H$;
        \item $GameDesign_{g,o}$: Sessione individuale di game design che si svolge il giorno $g \in G$ nella fascia oraria $h \in H$;
        \item $Grafica_{g,o}$: Sessione individuale di grafica che si svolge il giorno $g \in G$ nella fascia oraria $h \in H$;
        \item $guadagnoProgrammazione$: Guadagno derivato dallo svolgimento di una sessione di programmazione;
        \item $guadagnoGameDesign$: Guadagno derivato dallo svolgimento di una sessione di game design;
        \item $guadagnoGrafica$: Guadagno derivato dallo svolgimento di una sessione di grafica;
        \item $minSessioniAvanzate$: Numero minimo di sessioni avanzate che devono essere svolte alla settimana;
        \item $orarioSessioneAvanzata$: Fascia oraria in cui si svolgono le sessioni avanzate;
        \item $guadagnoSessioneAvanzata$: Guadagno derivato dallo svolgimento di una sessione avanzata;
        \item $oreSessioneSpeciale$: Numero di ore di una sessione speciale;
        \item $guadagnoSessioneSpeciale$: Guadagno derivato dallo svolgimento di una sessione speciale.
        \item $M$: costante di grandi dimensioni.
    \end{itemize}
   \subsection{Variabili decisionali}
   \begin{itemize}
    \item $p_{e,g,h}$: Variabile binaria che vale 1 se l'esperto $e \in E$ segue il corso di programmazione $g \in G$ il giorno $h \in H$;
    \item $d_{e,g,h}$: Variabile binaria che vale 1 se l'esperto $e \in E$ segue il corso di game design $g \in G$ il giorno $h \in H$;
    \item $r_{e,g,h}$: Variabile binaria che vale 1 se l'esperto $e \in E$ segue il corso di grafica $g \in G$ il giorno $h \in H$;
    \item $t_g$: Variabile binaria che vale 1 se nel giorno $g \in G$ si tiene una sessione avanzata;
    \item $a_{e,g}$: Variabile binaria che vale 1 se l'esperto $e \in E$ segue la sessione avanzata il giorno $g \in G$;
    \item $s$: Variabile binaria che vale 1 se nel fine settimana si tiene una sessione speciale;
    \item $b_{e}$: Variabile binaria che vale 1 se l'esperto $e \in E$ segue la sessione speciale.
   \end{itemize}
   \subsection{Funzione obiettivo}
    $\max$ \textbf{Guadagno Programmazione} $+$ \textbf{Guadagno Game design} $+$ \textbf{Guadagno Grafica} $+$      \textbf{Guadagno Avanzata} $+$ \textbf{Guadagno Speciale} $-$ \textbf{Costo degli esperti}\\
    \newline
   dove i termini sono:\\ \\
    \textbf{Guadagno Programmazione}: $\sum_{g \in G, h \in H} Programmazione_{g,h} \cdot guadagnoProgrammazione$;\\ \\
    \textbf{Guadagno Game design}: $\sum_{g \in G, h \in H} GameDesign_{g,h} \cdot guadagnoGameDesign$;\\ \\
    \textbf{Guadagno Grafica}: $\sum_{g \in G, h \in H} Grafica_{g,h} \cdot guadagnoGrafica$;\\ \\
    \textbf{Guadagno Avanzata}: $\sum_{g \in G} t_g \cdot guadagnoSessioneAvanzata$;\\ \\
    \textbf{Guadagno Speciale}: $s \cdot guadagnoSessioneSpeciale$;\\ \\
    \textbf{Costo degli esperti}:\\ $C_e \cdot ($ \textbf{Ore sessioni individuali} $+$ \textbf{Ore sessioni avanzate} $+$ \textbf{Ore sessioni speciali}$)$
    \begin{itemize}
        \item \textbf{Ore sessioni individuali}: $\sum_{e \in E, g \in G, h \in H} (p_{e,g,h} + d_{e,g,h} + r_{e,g,h})$;
        \item \textbf{Ore sessioni avanzate}: $\sum_{e \in E, g \in G} a_{e,g} \cdot 4$;
        \item \textbf{Ore sessioni speciali}: $\sum_{e \in E} b_{e} \cdot oreSessioneSpeciale$.
    \end{itemize}
    \newpage
    \textbf{S.T.}\\
    \begin{itemize}
        \item Ogni corso deve avere almeno un istrurre che lo segue:
    \end{itemize}
    \begin{tabular*}{\textwidth}{@{\extracolsep{\fill}} ll}
        $\sum_{e \in E} p_{e,g,h} \cdot M \geq Programmazione_{g,h}$ & $\forall g \in G, \forall h \in H$ \\
        & \\
        $\sum_{e \in E} d_{e,g,h} \cdot M \geq GameDesign_{g,h}$ & $\forall g \in G, \forall h \in H$ \\
        & \\
        $\sum_{e \in E} r_{e,g,h} \cdot M \geq Grafica_{g,h}$ & $\forall g \in G, \forall h \in H$ \\
        & \\   
        $\sum_{e \in E} a_{e,g} \geq t_g$ & $\forall g \in G$ \\
        & \\
        $\sum_{e \in E} b_{e} \geq s$ & \\
    \end{tabular*}
    \begin{itemize}
        \item Un esperto non può seguire più sessioni di tipo diverso contemporaneamente:
    \end{itemize}
    \begin{tabular*}{\textwidth}{@{\extracolsep{\fill}} ll}
        $p_{e,g,h} + d_{e,g,h} + r_{e,g,h} \leq 1$ & $\forall e \in E, \forall g \in G, \forall h \in H$ \\
    \end{tabular*}
    \begin{itemize}
        \item Un esperto può seguire un numero massimo di sessioni contemporaneamente:
    \end{itemize}
    \begin{tabular*}{\textwidth}{@{\extracolsep{\fill}} ll}
        $\sum_{e \in E} p_{e,g,h} \geq \frac{Programmazione_{g,h}}{maxSessioni_e}$ & $\forall g \in G, \forall h \in H$ \\
         & \\
        $\sum_{e \in E} d_{e,g,h} \geq \frac{GameDesign_{g,h}}{maxSessioni_e}$ & $\forall g \in G, \forall h \in H$ \\
            & \\
        $\sum_{e \in E} r_{e,g,h} \geq \frac{Grafica_{g,h}}{maxSessioni_e}$ & $\forall g \in G, \forall h \in H$ \\
    \end{tabular*}
    \begin{itemize}
        \item Un esperto può seguire un numero massimo di ore settimanali:
    \end{itemize}
    \begin{tabular*}{\textwidth}{@{\extracolsep{\fill}} ll}
        $(\sum_{g \in G, h \in H} p_{e,g,h} + d_{e,g,h} + r_{e,g,h}) + (\sum_{g \in G}4 \cdot a_{e,g}) + b_e \cdot oreSessioneSpeciale$ & $\forall i \in I$ \\
    \end{tabular*}
    \begin{itemize}
        \item Se un esperto segue una sessione avanzata deve seguire solo quello per tutte le 4 ore:
    \end{itemize}
    $p_{e,g,h} + p_{e,g,h+1} + p_{e,g,h+2} + p_{e,g,h+3} + d_{e,g,h} + d_{e,g,h+1} + d_{e,g,h+2} + d_{e,g,h+3}$ \newline $+ r_{e,g,h} + r_{e,g,h+1} + r_{e,g,h+2} + r_{e,g,h+3}$\\
    \begin{tabular*}{\textwidth}{@{\extracolsep{\fill}} ll}
        & $\forall e \in E, \forall g \in G, h = orarioSessioneAvanzata$ \\
    \end{tabular*}
    \begin{itemize}
        \item Gli sviluppatori senior devono seguire almeno un certo numero di sessioni avanzate alla settimana:
    \end{itemize}
    \begin{tabular*}{\textwidth}{@{\extracolsep{\fill}} ll}
        $\sum_{g \in G} t_g \geq minSessioniAvanzate$ & $\forall g \in G$ \\
    \end{tabular*}
    \\ \\ \\ \\
    \begin{tabular*}{\textwidth}{@{\extracolsep{\fill}} ll}
        \textbf{Dominio:} $p_{e,g,h},d_{e,g,h},r_{e,g,h},t_g,a_{e,g},s,b_e \in \{0,1\}$ & $\forall e \in E, \forall g \in G, \forall h \in H$ \\
    \end{tabular*}
    \newpage
    \section{Implementazione}
    \subsection{File .mod}
    \subsection{File .run}
    \section{Primo scenario}
    \subsection{File .dat}
    \subsection{Risultati}
    \section{Secondo scenario}
    \subsection{File .dat}
    \subsection{Risultati}
\end{document}
